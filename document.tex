\documentclass{article}
% Comment the following line to NOT allow the usage of umlauts
\usepackage[utf8]{inputenc}
% Uncomment the following line to allow the usage of graphics (.png, .jpg)
%\usepackage{graphicx}

% Start the document
\begin{document}

% Create a new 1st level heading
\section[pt100]{AIR-PURIFYING RESPIRATOR}








\subsection{introduction}









[pt50]An air purifier or air cleaner is a device which removes contaminants from the air in a room to improve indoor air quality. These devices are commonly marketed as being beneficial to allergy sufferers and asthmatics, and at reducing or eliminating second-hand tobacco smoke.


Gas mask also known as air purifier respirator are defined by regulator device designed for use during entry into atmosphere not immediately dangerous to life or health on escape only from hazardous atmosphere containing
adequate oxygen to support life.










\subsection{working}





Mechanical filter work by physically
trapping particulate matter.Unlike chemical filter,mechanical filter become
more efficient with use but make it harder to breathe.
Gas masks include a cartlidge or consist activated carbon and/or chemical
to remove dangerous gas and vapour contaminant.A particulate filter may
be attached to the cartridge or canister.















\subsection{uses}














manufacturing (automotive, chemical, metal fabrication, food and beverage,
wood working, paper and pulp), mining, construction, agriculture and forestry,
cement production, power generation, shipbuilding and the textile industry.
A gas mask is designed to protect the face and lungs against a noxious
gases and fumes , chemical agents and biological substance as in welfare
terror attack or in certain industrial environment.
Respirators require user training in order to provide proper protection.

\subsection{Advantages}







Air purifier respirator help to those people who were allergic to dust
and smog.A gas mask is designed to protect the face and lungs against a noxious
gases and fumes,chemical agents and biological substance as in welfare
terror attack or in certain industrial environment.
In city like Delhi there was a huge amount of air polluted ,air purifier
respirator helps to breathe safely and efficiently.
\clearpage



\subsection{Disadvantages}











Never use air purifier respirator at that place oxygen supply is weak These
respirator only purifier air not supply the oxygen.Air purifier respiratory don’t provide oxygen. If it is used in environment
with low oxygen level such as in a fire or a confined space,you are in a danger
of asphyxiation.
It provide no eye protection.
Provide no protection against irritant gases such as ammonia.


\subsection{conclusion}


Present contaminate don’t have warning properties. Many contaminate
have warning properties at high concentration.Detection of contaminant that
don’t have warning properties can be difficult and therefore contaminate can
be difficult and therefore can leak through or around a respirator and you
won’t know it.
The contaminate has a skin designation.Unless other PPE is also used
contaminant still result through dermal absorption can with an approved
respirator.Air purifying respirator can only protect at or below specific concentra-
tion of contaminant.
Air purifier respirator work by removing gas vapour, aerosol, or a combinationof contaminant from air through the use of filter,cartlidge or canister. These
respirator don’t supply oxygen and therfore cannot be used in an atmosphere
that is oxygen deficient or immediately dangerous to health.
\clearpage

\section{     MRI}













\subsection{introduction}




A non-invasive imaging technology used to investigate anatomy and function of the body in both health and disease without the use of damaging ionizing radiation. It is often used for disease detection, diagnosis, and treatment monitoring. It is based on sophisticated technology that excites and detects changes in protons found in the water that makes up living tissues. 















\subsection{working}

MRIs employ powerful magnets which produce a strong magnetic field that forces protons in the body to align with that field. When a radiofrequency current is then pulsed through the patient, the protons are stimulated, and spin out of equilibrium, straining against the pull of the magnetic field. When the radiofrequency field is turned off, the MRI sensors are able to detect the energy released as the protons realign with the magnetic field. The time it takes for the protons to realign with the magnetic field, as well as the amount of energy released, changes depending on the environment and the chemical nature of the molecules. Physicians are able to tell the difference between various types of tissues based on these magnetic properties

\subsection{uses}


MRIS scanners are particularly well suited to image the non-bony parts or soft tissues of the body. They differ from computed tomography (CT), in that they do not use the damaging ionizing radiation of x-rays. The brain, spinal cord and nerves, as well as muscles, ligaments, and tendons are seen much more clearly with MRI than with regular x-rays and CT; for this reason MRI is often used to image knee and shoulder injuries.

In the brain, MRI can differentiate between white matter and grey matter and can also be used to diagnose aneurysms and tumors. Because MRI does not use x-rays or other radiation, it is the imaging modality of choice when frequent imaging is required for diagnosis or therapy, especially in the brain. However, MRI is more expensive than x-ray imaging or CT scanning.
\clearpage

\subsection{advantages}
one kind of specialized MRI is functional Magnetic Resonance Imaging (fMRI.) This is used to observe brain structures and determine which areas of the brain “activate” (consume more oxygen) during various cognitive tasks. It is used to advance the understanding of brain organization and offers a potential new standard for assessing neurological status and neurosurgical risk.



\subsection{disadvantages}
MRI does not emit the ionizing radiation that is found in x-ray and CT imaging, it does employ a strong magnetic field. The magnetic field extends beyond the machine and exerts very powerful forces on objects of iron, some steels, and other magnetizable objects; it is strong enough to fling a wheelchair across the room. Patients should notify their physicians of any form of medical or implant prior to an MR scan.

People with implants, particularly those containing iron, — pacemakers, vagus nerve stimulators, implantable cardioverter- defibrillators, loop recorders, insulin pumps, cochlear implants, deep brain stimulators, and capsules from capsule endoscopy should not enter an MRI machine.


Noise—loud noise commonly referred to as clicking and beeping, as well as sound intensity up to 120 decibels in certain MR scanners, may require special ear protection.


Pregnancy—while no effects have been demonstrated on the fetus, it is recommended that MRI scans be avoided as a precaution especially in the first trimester of pregnancy when the fetus’ organs are being formed and contrast agents, if used, could enter the fetal bloodstream.
\clearpage

\section{COMPUTED TOMOGRAPHY (CT)}

\subsection{introduction}
The term “computed tomography”, or CT, refers to a computerized x-ray imaging procedure in which a narrow beam of x-rays is aimed at a patient and quickly rotated around the body, producing signals that are processed by the machine’s computer to generate cross-sectional images—or “slices”—of the body. These slices are called tomographic images and contain more detailed information than conventional x-rays. Once a number of successive slices are collected by the machine’s computer, they can be digitally “stacked” together to form a three-dimensional image of the patient that allows for easier identification and location of basic structures as well as possible tumors or abnormalities.


\subsection{working}

Unlike a conventional x-ray—which uses a fixed x-ray tube—a CT scanner uses a motorized x-ray source that rotates around the circular opening of a donut-shaped structure called a gantry. During a CT scan, the patient lies on a bed that slowly moves through the gantry while the x-ray tube rotates around the patient, shooting narrow beams of x-rays through the body. Instead of film, CT scanners use special digital x-ray detectors, which are located directly opposite the x-ray source. As the x-rays leave the patient, they are picked up by the detectors and transmitted to a computer.

Each time the x-ray source completes one full rotation, the CT computer uses sophisticated mathematical techniques to construct a 2D image slice of the patient. The thickness of the tissue represented in each image slice can vary depending on the CT machine used, but usually ranges from 1-10 millimeters. When a full slice is completed, the image is stored and the motorized bed is moved forward incrementally into the gantry. The x-ray scanning process is then repeated to produce another image slice. This process continues until the desired number of slices is collected.

\subsection{uses}

CT scans can be used to identify disease or injury within various regions of the body. For example, CT has become a useful screening tool for detecting possible tumors or lesions within the abdomen. A CT scan of the heart may be ordered when various types of heart disease or abnormalities are suspected. CT can also be used to image the head in order to locate injuries, tumors, clots leading to stroke, hemorrhage, and other conditions. It can image the lungs in order to reveal the presence of tumors, pulmonary embolisms (blood clots), excess fluid, and other conditions such as emphysema or pneumonia. A CT scan is particularly useful when imaging complex bone fractures, severely eroded joints, or bone tumors since it usually produces more detail than would be possible with a conventional x-ray.
\clearpage

\subsection{advantages}

with all x-rays, dense structures within the body—such as bone—are easily imaged, whereas soft tissues vary in their ability to stop x-rays and, thus, may be faint or difficult to see. For this reason, intravenous (IV) contrast agents have been developed that are highly visible in an x-ray or CT scan and are safe to use in patients. Contrast agents contain substances that are better at stopping x-rays and, thus, are more visible on an x-ray image. For example, to examine the circulatory system, a contrast agent based on iodine is injected into the bloodstream to help illuminate blood vessels. This type of test is used to look for possible obstructions in blood vessels, including those in the heart. Oral contrast agents, such as barium-based compounds, are used for imaging the digestive system, including the esophagus, stomach, and GI tract.


\subsection{disadvantages}


CT scans can diagnose possibly life-threatening conditions such as hemorrhage, blood clots, or cancer. An early diagnosis of these conditions could potentially be life-saving. However, CT scans use x-rays, and all x-rays produce ionizing radiation. Ionizing radiation has the potential to cause biological effects in living tissue. This is a risk that increases with the number of exposures added up over the life of an individual. However, the risk of developing cancer from radiation exposure is generally small.

A CT scan in a pregnant woman poses no known risks to the baby if the area of the body being imaged isn’t the abdomen or pelvis. In general, if imaging of the abdomen and pelvis is needed, doctors prefer to use exams that do not use radiation, such as MRI or ultrasound. However, if neither of those can provide the answers needed, or there is an emergency or other time constraint, CT may be an acceptable alternative imaging option.
\clearpage

\section{DIAGNOSTIC ULTRASOUND}

\subsection{introduction}
Diagnostic ultrasound is a non-invasive diagnostic technique used to image inside the body. Ultrasound probes, called transducers, produce sound waves that have frequencies above the threshold of human hearing (above 20KHz), but most transducers in current use operate at much higher frequencies (in the megahertz (MHz) range). Most diagnostic ultrasound probes are placed on the skin. However, to optimize image quality, probes may be placed inside the body via the gastrointestinal tract, vagina, or blood vessels. In addition, ultrasound is sometimes used during surgery by placing a sterile probe into the area being operated on.  
Diagnostic ultrasound can be further sub-divided into anatomical and functional ultrasound. Anatomical ultrasound produces images of internal organs or other structures. Functional ultrasound combines information such as the movement and velocity of tissue or blood, softness or hardness of tissue, and other physical characteristics, with anatomical images to create “information maps.” These maps help doctors visualize changes/differences in function within a structure or organ.

\subsection{working}
ultrasound waves are produced by a transducer, which can both emit ultrasound waves, as well as detect the ultrasound echoes reflected back. In most cases, the active elements in ultrasound transducers are made of special ceramic crystal materials called piezoelectrics. These materials are able to produce sound waves when an electric field is applied to them, but can also work in reverse, producing an electric field when a sound wave hits them. When used in an ultrasound scanner, the transducer sends out a beam of sound waves into the body. The sound waves are reflected back to the transducer by boundaries between tissues in the path of the beam (e.g. the boundary between fluid and soft tissue or tissue and bone). When these echoes hit the transducer, they generate electrical signals that are sent to the ultrasound scanner. Using the speed of sound and the time of each echo’s return, the scanner calculates the distance from the transducer to the tissue boundary. These distances are then used to generate two-dimensional images of tissues and organs.

\subsection{uses}


Diagnostic ultrasound is able to non-invasively image internal organs within the body. However, it is not good for imaging bones or any tissues that contain air, like the lungs. Under some conditions,ultrasound can image bones (such as in a fetus or in small babies) or the lungs and lining around the lungs, when they are filled or partially filled with fluid. One of the most common uses of ultrasound is during pregnancy, to monitor the growth and development of the fetus, but there are many other uses, including imaging the heart, blood vessels, eyes, thyroid, brain, breast,abdominal organs, skin, and muscles




\subsection{advantages}

Ultrasound is also an important method for imaging interventions in the body. For example, ultrasound-guided needle biopsy helps physicians see the position of a needle while it is being guided to a selected target, such as a mass or a tumor in the breast. Also, ultrasound is used for real-time imaging of the location of the tip of a catheter as it is inserted in a blood vessel and guided along the length of the vessel. It can also be used for minimally invasive surgery to guide the surgeon with real-time images of the inside of the body.
\subsection{disadvantages}
Diagnostic ultrasound is generally regarded as safe and does not produce ionizing radiation like that produced by x-rays. Still, ultrasound is capable of producing some biological effects in the body under specific settings and conditions. For this reason, the FDA requires that diagnostic ultrasound devices operate within acceptable limits. The FDA, as well as many professional societies, discourage the casual use of ultrasound (e.g. for keepsake videos) and recommend that it be used only when there is a true medical need.
\clearpage

\section{X-RAYS}

\subsection{introduction}

A form of high energy electromagnetic radiation that can pass through most objects, including the body. X-rays travel through the body and strike an x-ray detector (such as radiographic film, or a digital x-ray detector) on the other side of the patient, forming an image that represents the “shadows” of objects inside the body. 

\subsection{working}

To create a radiograph, a patient is positioned so that the part of the body being imaged is located between an x-ray source and an x-ray detector. When the machine is turned on, x-rays travel through the body and are absorbed in different amounts by different tissues, depending on the radiological density of the tissues they pass through. Radiological density is determined by both the density and the atomic number (the number of protons in an atom’s nucleus) of the materials being imaged. For example, structures such as bone contain calcium, which has a higher atomic number than most tissues. Because of this property, bones readily absorb x-rays and, thus, produce high contrast on the x-ray detector. As a result, bony structures appear whiter than other tissues against the black background of a radiograph. Conversely, x-rays travel more easily through less radiologically dense tissues such as fat and muscle, as well as through air-filled cavities such as the lungs. These structures are displayed in shades of gray on a radiograph.
\subsection{uses}

Listed below are examples of examinations and procedures that use x-ray technology to either diagnose or treat disease:

Diagnostic
X-ray radiography: Detects bone fractures, certain tumors and other abnormal masses, pneumonia, some types of injuries, calcifications, foreign objects, dental problems, etc.

Mammography: A radiograph of the breast that is used for cancer detection and diagnosis. Tumors tend to appear as regular or irregular-shaped masses that are somewhat brighter than the background on the radiograph.

computed tomography): Combines traditional x-ray technology with computer processing to generate a series of cross-sectional images of the body that can later be combined to form a three-dimensional x-ray image.

Fluoroscopy: Uses x-rays and a fluorescent screen to obtain real-time images of movement within the body or to view diagnostic processes, such as following the path of an injected or swallowed contrast agent. For example, fluoroscopy is used to view the movement of the beating heart, and, with the aid of radiographic contrast agents, to view blood flow to the heart muscle as well as through blood vessels and organs

therapeutic

Radiation therapy in cancer treatment: X-rays and other types of high-energy radiation can be used to destroy cancerous tumors and cells by damaging their DNA. The radiation dose used for treating cancer is much higher than the radiation dose used for diagnostic imaging.

\subsection{disadvantages}

When used appropriately, the diagnostic benefits of x-ray scans significantly outweigh the risks. X-ray scans can diagnose possibly life-threatening conditions such as blocked blood vessels, bone cancer, and infections. However, x-rays produce ionizing radiation—a form of radiation that has the potential to harm living tissue. This is a risk that increases with the number of exposures added up over the life of the individual. However, the risk of developing cancer from radiation exposure is generally small.

An x-ray in a pregnant woman poses no known risks to the baby if the area of the body being imaged isn’t the abdomen or pelvis. In general, if imaging of the abdomen and pelvis is needed, doctors prefer to use exams that do not use radiation, such as MRI or ultrasound. However, if neither of those can provide the answers needed, or there is an emergency or other time constraint, an x-ray may be an acceptable alternative imaging option.

Children are more sensitive to ionizing radiation and have a longer life expectancy and, thus, a higher relative risk for developing cancer than adults. Parents may want to ask the technologist or doctor if their machine settings have been adjusted for children.


% Uncomment the following two lines if you want to have a bibliography
%\bibliographystyle{alpha}
%\bibliography{document}

\end{document}
